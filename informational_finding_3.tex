\section{Finding - Outdated Sudo Version}
\hrule\begin{table}[htb]
    \renewcommand{\arraystretch}{1.5}
    \begin{tabular*}{\textwidth}{|>{\columncolor{yellow!15}}p{3cm}|p{17.2cm}|}
    \textbf{Finding} & \textbf{Outdated Sudo version}\\
    Risk& Low\\
    Category& Patching\\
    Impact& \\\\
    Description& Executing the command: ''sudo -v'' displayed the sudo version 1.9.5p2 that the system is using. Although the installed version of sudo on the DUT is stable and has been available for some time, it is not the most recent version. As newer versions of sudo may have significant bug fixes, it is recommended to update to the latest version. While sudo 1.9.5p2 has fixed a critical vulnerability related to a Heap-based Buffer Overflow, it is still advisable to upgrade to the latest version.
	\\ 
	&\\
	&\\
	&\\
	&\\
	&\\
	&\\
	&\\
	&\\
    Recommendation& To mitigate the vulnerabilities present in sudo, it is highly advisable to upgrade to the latest version of the software, which is free from such weaknesses. The official sudo website at sudo.ws offers the most recent stable releases of sudo. By updating to the latest secure version, users can effectively address known vulnerabilities and bugs, thereby enhancing the overall security and stability of their systems.\\ 
    \\\\\\\\\\\\\\\\   
    \end{tabular*}
    \end{table}