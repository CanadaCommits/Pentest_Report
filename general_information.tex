\chapter{General Information}

\section*{Risk Assessment}
The following risk assessment is based on both personal experience and objective fact. The aim of this assessment is to provide the client with an overview of the potential risks associated with their information technology (IT) infrastructure. Through this evaluation, we hope to identify the most significant risks and potential consequences of those risks, enabling the client to make informed decisions on how to mitigate those risks.
\\\\
Our assessment is based on a combination of personal experience and best practices in the field of IT security. We have conducted extensive research and analysis, examining the client's IT infrastructure and identifying any vulnerabilities or weaknesses. 
\\\\
The purpose of this assessment is to give the client an idea of the severity of the risks present in their IT infrastructure. By providing a clear understanding of the potential consequences of these risks, the client can prioritize their resources to address the most significant risks first. It is important to note that while we have taken every effort to identify and evaluate all potential risks, new threats and vulnerabilities can arise at any time. Therefore, this risk assessment should be considered an ongoing process and reviewed regularly to ensure its relevance and accuracy.

\section*{Risk Matrix}
The following matrix was used to categorize the risk level of the vulnerabilities listed in this document:
\\

\begin{table}[htb]
    \renewcommand{\arraystretch}{1.5}
    \begin{tabular*}{\textwidth}{|p{3.3cm}|p{3cm}|p{3cm}|p{3cm}|}
        \hline
        Impact/Likelihood&Low&Medium &High\\
    \hline
    &\cellcolor{red!30}High&\cellcolor{red!30}High&\cellcolor{red}Critical\\
    \hline
    Medium&Low& Medium&Medium \\
    \hline
    Low&Low&Low&Low\\  
    \hline
    \end{tabular*}
    \end{table}    
