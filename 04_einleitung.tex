\chapter{Einleitung}
\label{sec:einleitung}

\section{Problemstellung}
Die Abteilung WebApps innerhalb des Unternehmens MLP Finanzberatung SE ist verantwortlich für die Entwicklung von Web-Applikation für die Finanzberater des Unternehmens.
Die Sicherheit dieser System spielt dabei eine entscheidene Rolle. Ist es Angreifern möglich in die Systeme einzudringen könnte dies, je nach Angriff, bedeuten, dass sensible Daten gestohlen werden könnten oder gewisse Services für die Mitarbeiter nicht mehr zur Verfügung stehen würden.
Vor diesem Hintergrund ist es unabdingbar seine Web-Applikation mithilfe von Penetration Testing auf Sicherheitslücken zu untersuchen.
Zum Einsatz kommen hierbei verschiedene Software-Produkte und das Betriebsystem Kali Linux, welches auf Penetration Testing ausgerichtet ist.

\section{Zielsetzung}
Das Ziel dieser Arbeit ist es daher sich in die Rolle des Angreifers hineinzuversetzen und verschiedene Angriffsmethoden zu erlernen. Dafür ist es nötig sich Hintergrundswissen unterschiedlicher Schwachstellen anzueignen und Software zu erlernen, die für Website Penetration Testing eingesetzt wird.

\section{Vorgehensweise}
In dieser Arbeit werden zuerst Begriffe verschiedene Begriffe, wie Ethical Hacking zu erläutert.
Zudem werden verschiedene Schwachstellen, die innerhalb der Web-Entwicklung auftreten können definiert, aufgezeigt wie diese ausgenutzt werden können und passende Gegenmaßnahmen diskutiert. Des Weiteren werden Software-Lösungen zum Analysieren von Websiten auf Sicherheitslücken vorgestellt.

