
\begin{table}[htb]
    \renewcommand{\arraystretch}{1.5}
    \begin{tabular*}{\textwidth}{|>{\columncolor{red!15}}p{3cm}|p{17.3cm}|}
    \textbf{\large Finding} & \textbf{\large userconf-pi usage}\section*{}\addcontentsline{toc}{section}{Finding 18 - ''userconf-pi'' usage}
    \\
    Risk& High\\
    Category& Misconfiguration\\
    Impact& An attacker can modify the passwords of the users.\\\\
    Description& The DUT is equipped with the userconf-pi tool, which presents an interactive configuration menu on its first bootup with a display interface. The menu offers various options for user customization, including the ability to change the usernames of existing accounts. Additionally, users can modify the password for the selected account after the username has been changed. Consequently, changing a password can be easily accomplished by connecting a display to the DUT and initiating the first boot.
	\\ 
	&\\
	&\\
    Recommendation& To ensure security and prevent users from changing any password upon the first boot, it is highly advised to uninstall the userconf-pi tool from the DUT via the apt package manager. However, if the tool is necessary, it's recommended to disable the feature that permits password changes upon the first boot by adjusting the relevant settings.\\    
    \\\\\\\\\\\\\\\\\\\\\\\\\\\\\\\\\\\\\\\\\\\\\\\\\\\\
    \end{tabular*}
    \end{table}