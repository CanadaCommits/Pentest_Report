\begin{table}[htb]
    \renewcommand{\arraystretch}{1.5}
    \begin{tabular*}{\textwidth}{|>{\columncolor{red!30}}p{3cm}|p{17.3cm}|}
    \textbf{\large Finding} & \textbf{\large Root OpenSSL access through management server}\section*{}\addcontentsline{toc}{section}{Finding 15 - Root OpenSSL access through management server}
    \\
    Risk& Critical\\
    Category& Access Controls, Obfuscation\\
    Impact& An attacker can gain root access of the OpenSSH server.\\\\
    Description& While scanning for open ports of the DUT a service running on port 20321 was found. After detailed examination the service was identified to be a OpenSSL server setup in python. The file  ''check\_version.pyc'' was found by displaying the currently running processes using the command: ''ps aux'' Reviewing the python script of attachment \ref{sec:attachment2} an insufficient certificate validation was noticed:

    \begin{lstlisting}[language=python]
elif self . _client_cert == " subject = CN = Management Client Certificate , O = Secure Systems Inc . , OU = admin = true " :
    print ( " Admin user logged in " )
    print ( " Administrator access granted . " , file = self . _proc . stdin )
else :
    print ( " Invalid certificate found " )\end{lstlisting}
    The certificate validation only checks the subject of the certificate but not the issuer. Given this information it was possible to create our own certificate and key matching the parameters validated:

    \begin{lstlisting}[language=bash]
openssl req -x509 -newkey rsa:4096 -keyout client.key -out client.crt -days 365 -subj
"/CN=Management Client Cer=ficate/O=Secure Systems Inc./OU=admin=true"\end{lstlisting}

Now a client can connect to the OpenSSL server by passing the certificate as a parameter and is logged in as admin:

\begin{lstlisting}[language=bash]
openssl s_client -connect 172.16.0.29:20321 -key client.key -cert client.crt\end{lstlisting}
	\\ 
	&\\
	&\\
    Recommendation& In order to maintain secure communication between the management server and the client, it is crucial to carry out appropriate validation of the client certificate. This involves multiple steps such as validating the certificate's authenticity and ensuring that it is signed by a trusted certificate authority. Moreover, the management server must verify whether the certificate has been revoked or if it is currently expired or invalid. By conducting these checks, the management server can guarantee that only legitimate clients with valid certificates are permitted to communicate with the server.\\\\\\\\\\\\\\\\\\\\\\\\\\\\\\\\\\\\\\\\\\\\\\\\\\\\\\\\\\\\\\
    \end{tabular*}
    \end{table}