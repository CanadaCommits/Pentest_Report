\section{Finding 17 - Remote Code Execution through vulnerable software}
\hrule\begin{table}[htb]
    \renewcommand{\arraystretch}{1.5}
    \begin{tabular*}{\textwidth}{|>{\columncolor{red!15}}p{3cm}|p{17.2cm}|}
    \textbf{Finding} & \textbf{}\\
    Risk& High\\
    Category& Remote Code Execution\\
    Impact& An attacker can execute shell commands remotely\\\\
    Description& This script is vulnerable to a command injection attack, which allows an attacker to execute arbitrary commands on the DUT. The vulnerability arises due to the script's use of user-controlled input as part of a shell command without proper input validation.

    The script sends a GET request to a remote server with the MAC-Address of the DUT as an argument. If the server responds with a 200 status code, the script executes the response arguments in a shell on the DUT. The attacker can craft a malicious response that includes arbitrary shell commands, which will then be executed on the DUT.
    The following command retrieves the MAC-Address of the network interface "eth0" from the DUT and includes it as an argument in a GET request to the server (/mac=MAC-Address):
    ''ip link show eth0''

	\\ 
	&\\
	&\\
	&\\
	&\\
	&\\
	&\\
	&\\
	&\\
    Recommendation&  To fix this vulnerability, the script should validate and sanitize the input before using it in a shell command. One way to achieve this is to use an appropriate library or function to escape any shell metacharacters in the input before using it in a shell command. Additionally, the script should limit the allowed characters and length of the input to only what is necessary for the intended functionality.\\    
    \end{tabular*}
    \end{table}